\subsection{Non-Hermitian Schrödinger-Pauli Mapping}

\subsubsection{Background}

Meng \& Yang (2024) demonstrated that the incompressible Navier-Stokes equations can be mapped to a non-Hermitian Schrödinger-Pauli equation for a quantum spin system. This mapping provides the mathematical framework for understanding NS turbulence as suppressed quantum dynamics with an essential dissipative component.

\subsubsection{The Navier-Stokes Equations}

For an incompressible fluid:
\begin{align}
\partial_t \mathbf{u} + (\mathbf{u} \cdot \nabla)\mathbf{u} &= -\nabla p/\rho + \nu \nabla^2 \mathbf{u} \\
\nabla \cdot \mathbf{u} &= 0
\end{align}
where $\mathbf{u}$ is velocity field, $p$ is pressure, $\rho$ is density, and $\nu$ is kinematic viscosity.

\subsubsection{The Madelung Transformation}

Define a complex wave function from the velocity field:
\begin{equation}
\psi(\mathbf{x},t) = \sqrt{\rho} \exp(iS/\hbar_{\text{eff}})
\end{equation}
where $S$ is the velocity potential ($\mathbf{u} = \nabla S$) and $\hbar_{\text{eff}}$ is an effective Planck constant.

The probability density and current are:
\begin{align}
\rho &= |\psi|^2 \\
\mathbf{j} &= \frac{\hbar_{\text{eff}}}{m^*} \text{Im}(\psi^* \nabla \psi)
\end{align}

Starting from the spinor-velocity relation in the Madelung representation:
\begin{equation}
\mathbf{u} \simeq \frac{\hbar_{\text{eff}}}{m^*}\nabla\theta \implies |\nabla\psi|^2 \sim \frac{m^{*2}}{\hbar_{\text{eff}}^2}\rho|\mathbf{u}|^2
\end{equation}
where $\hbar_{\text{eff}} = \nu$ and $m^* = \rho$ in the Meng-Yang dictionary.

For the Kerr horizon with $u \sim \Omega_H r_+$:
\begin{equation}
V_I^{(\text{Kerr})} = \nu\frac{m^{*2}}{\hbar_{\text{eff}}^2}\rho(\Omega_H r_+)^2 = \nu\rho\frac{a^2 r_+^2}{(r_+^2 + a^2)^2}
\end{equation}
This scaling confirms $\overline{V_I} \neq 0$ for all $J \neq 0$.

\subsubsection{The Resulting Schrödinger-Pauli Equation}

The NS equations transform to:
\begin{equation}
i\hbar_{\text{eff}} \partial_t \psi = \hat{H}\psi
\end{equation}
where the Hamiltonian is:
\begin{equation}
\hat{H} = -\frac{\hbar^2_{\text{eff}}}{2m^*} \nabla^2 + V_R(\mathbf{x}) + iV_I(\mathbf{x}) + \frac{\hbar_{\text{eff}}}{2}\sigma \cdot \mathbf{B}(\mathbf{x})
\end{equation}

\textbf{The parameters:}
\begin{itemize}
\item $\hbar_{\text{eff}} = \nu$ (kinematic viscosity serves as effective Planck constant)
\item $m^* = \rho$ (fluid density as effective mass)
\item $V_R(\mathbf{x})$ = real potential from pressure gradients
\item $V_I(\mathbf{x})$ = imaginary potential representing dissipation
\item $\sigma \cdot \mathbf{B}(\mathbf{x})$ = Stern-Gerlach term from vorticity
\end{itemize}

\subsubsection{Key Distinction - Hermiticity}

\begin{center}
\begin{tabular}{|l|c|c|c|c|}
\hline
\textbf{Flow Type} & \textbf{Viscosity} & $V_I$ & \textbf{Hermiticity} & \textbf{Reversibility} \\
\hline
Potential & $\nu = 0$ & 0 & Hermitian & Reversible \\
Euler & $\nu = 0$ & 0 & Hermitian & Reversible \\
\textbf{Navier-Stokes} & $\nu \neq 0$ & $\neq 0$ & \textbf{Non-Hermitian} & \textbf{Irreversible} \\
\hline
\end{tabular}
\end{center}

\subsubsection{The Stern-Gerlach Term Explicitly}

The effective magnetic field is the vorticity:
\begin{equation}
\mathbf{B}(\mathbf{x}) = \nabla \times \mathbf{u} = \omega
\end{equation}

The interaction term:
\begin{equation}
\hat{H}_{SG} = \frac{\hbar_{\text{eff}}}{2}\sigma \cdot \mathbf{B} = \frac{\hbar_{\text{eff}}}{2}(\sigma_x \omega_x + \sigma_y \omega_y + \sigma_z \omega_z)
\end{equation}
where $\sigma_i$ are the Pauli spin matrices:
\begin{equation}
\sigma_x = \begin{pmatrix} 0 & 1 \\ 1 & 0 \end{pmatrix}, \quad
\sigma_y = \begin{pmatrix} 0 & -i \\ i & 0 \end{pmatrix}, \quad
\sigma_z = \begin{pmatrix} 1 & 0 \\ 0 & -1 \end{pmatrix}
\end{equation}

\subsubsection{Physical Interpretation}

The Stern-Gerlach term creates a torque on the ``spin'' (local circulation element) analogous to a magnetic moment in an external field:
\begin{equation}
\tau = \mu \times \mathbf{B}_{\text{eff}}
\end{equation}

This torque drives precession and energy dissipation through radiation.

\subsubsection{Application to Euclidean Kerr Horizon}

For a rotating black hole with angular momentum $J$:

\begin{enumerate}
\item \textbf{Frame-dragging creates vorticity:}
\begin{equation}
\omega \sim g_{t\phi,r} \sim J/r^3 \text{ (near horizon)}
\end{equation}

\item \textbf{This appears as effective field:}
\begin{equation}
\mathbf{B}_{\text{eff}} = \omega \sim \Omega_H \text{ (at horizon)}
\end{equation}

\item \textbf{Creates Stern-Gerlach torque:}
\begin{equation}
\hat{H}_{SG} \sim \hbar_{\text{eff}} \Omega_H \sigma \cdot \hat{\phi}
\end{equation}

\item \textbf{Requires dissipation:}
The torque cannot be eliminated in closed Euclidean time $\tau$ without energy dissipation, which requires $V_I \neq 0$.
\end{enumerate}

\subsubsection{The Contradiction with Euclidean Signature}

In Euclidean $(+,+,+,+)$ with periodic $\tau$:
\begin{itemize}
\item Time dimension is compact: $\tau \in [0, \beta]$
\item System must be Hermitian (no coupling to external bath)
\item Hermitian $\Rightarrow V_I = 0$ (no dissipation)
\item But NS flow with rotation requires $V_I \neq 0$
\end{itemize}

\subsubsection{Resolution}

The signature must flip to Minkowski $(-,+,+,+)$ to:
\begin{enumerate}
\item Open the time dimension ($\tau \to t$, non-periodic)
\item Allow non-Hermitian dynamics ($V_I \neq 0$)
\item Enable temporal dissipation channel
\item Permit radiation to carry away angular momentum
\end{enumerate}

\subsubsection{Mathematical Statement}

The non-Hermiticity of the NS-spinor mapping is the \textbf{mathematical signature} of the physical necessity for temporal opening. The imaginary potential $V_I$ represents the coupling to the temporal dissipation channel that must exist when $J \neq 0$.

This provides the formal connection between:
\begin{itemize}
\item Geometric regularity (absence of conical singularity)
\item Non-Hermitian dynamics ($V_I \neq 0$)
\item Signature flip (opening time dimension)
\item Physical dissipation (radiation)
\end{itemize}

All four are different descriptions of the same requirement.