% ArXiv preprint template
\documentclass[11pt]{article}
\usepackage[utf8]{inputenc}
\usepackage{amsmath,amssymb,amsthm}
\usepackage{physics}
\usepackage{hyperref}
\usepackage{geometry}
\geometry{margin=1in}

\title{Signature Emergence from Rotational Stress: A Non-Wick Mechanism}
\author{Adam Morgan\\
\small Unaffiliated}
\date{\today}

\begin{document}

\maketitle

\begin{abstract}
We propose a physical mechanism for the emergence of Minkowski signature from Euclidean geometry based on the incompatibility of rotation with closed time dimensions. By establishing an equivalence between the Euclidean regularity condition and the requirement for temporal dissipation, we provide a non-Wick interpretation of signature selection. The Stern-Gerlach\cite{SternGerlach1922} spin alignment process serves as an observable analog, while recent mappings between Navier-Stokes\cite{NavierStokesMillennium} equations and non-Hermitian quantum spin systems provide the mathematical framework.  We prove that rotation-induced non-Hermiticity violates Reflection Positivity\cite{OsterwalderSchrader1973}, creating a Monodromy Obstruction that makes a smooth Euclidean thermal state mathematically inconsistent and necessitates the emergence of a Minkowski signature . This work connects the difficulty of the Navier-Stokes\cite{NavierStokesMillennium} millennium problem to fundamental questions about spacetime signature.
\end{abstract}

\section{Introduction}

\subsection{The Ontological Problem of Wick Rotation}

The Wick rotation $(t \to i\tau)$ is ubiquitous in quantum field theory and statistical mechanics, yet its physical meaning remains unclear. While mathematically convenient for path integral convergence, the transformation from Minkowski $(-,+,+,+)$ to Euclidean $(+,+,+,+)$ signature is typically treated as a purely analytical tool.

\textbf{Key Questions:}
\begin{enumerate}
\item What does ``imaginary time'' mean physically?
\item Why does the Wick rotation work?
\item Can signature selection be derived from physical principles?
\end{enumerate}

\subsection{Established Results}

\paragraph{From Black Hole Thermodynamics (Gibbons-Hawking\cite{GibbonsHawking1977}):}
\begin{itemize}
\item The Euclidean Kerr metric\cite{Kerr1963} has a conical singularity at the horizon
\item Regularity requires $\tau$-periodicity: $\beta = 4\pi/\kappa = 1/T_H$
\item This determines Hawking temperature
\end{itemize}

\paragraph{From Fluid/Gravity Correspondence\cite{FluidGravity2008}:}
\begin{itemize}
\item Black hole horizons map to viscous fluid dynamics
\item Kerr rotation induces vorticity in the dual fluid
\item Navier-Stokes\cite{NavierStokesMillennium} (NS) equations govern near-horizon dynamics
\end{itemize}

\paragraph{Recent Development (Meng \& Yang 2024):}
\begin{itemize}
\item NS equations map to non-Hermitian Schrödinger-Pauli equation
\item Mapping includes ``Stern-Gerlach\cite{SternGerlach1922} term'' $\sigma \cdot B$
\item Classical fluid interpreted as quantum spin system with imaginary diffusion
\end{itemize}
\subsection{The Challenge of Rotational Stress}

The established thermal interpretation of Euclidean gravity, as exemplified by the Gibbons--Hawking calculation, depends fundamentally on a smooth, static manifold where the Euclidean time $\tau$ is periodic and unmixed with spatial coordinates \cite{GibbonsHawking1977}. This success does not, however, address the deeper ontological question of signature selection when a system possesses internal stress.

For a rotating black hole (Kerr metric\cite{Kerr1963}), the Euclidean time coordinate $\tau$ becomes necessarily mixed with the azimuthal angle $\phi$ via the frame-dragging effect \cite{appendix_a1_kerr_expansion.tex}. This leads to the fundamental problem: \textbf{rotation-induced stress cannot dissipate within a closed, compact time dimension}. In the dual fluid description (via the Fluid/Gravity correspondence \cite{FluidGravity2008}), the rotational stress is mapped to \textbf{viscous vorticity ($\mathbf{\omega} = \nabla \times \mathbf{u}$)} near the horizon \cite{appendix_a2_vorticity_calculation.tex}.

If a rotating black hole must settle into a smooth thermal state with period $\beta$, then the system, when described in Euclidean time, must be \textbf{time-reversible and non-dissipative} (Hermitian). However, the rotational stress is fundamentally dissipative (viscous) and requires a channel to carry away angular momentum and energy \cite{Hawking1975}. The topological constraint of closed Euclidean time prevents this dissipation, creating an internal mathematical contradiction:
\[
    \text{Periodic Time } + \text{Rotation} + \text{Smoothness} \implies \text{Contradiction}
\]
To explore the mathematical nature of this constraint, we require a framework that can precisely link gravitational rotation to quantum dissipation. The key to our approach is the non-Hermitian representation of fluid dynamics.

\subsection{This Work}

We demonstrate that the incompatibility of rotation with closed Euclidean time is not merely a conceptual analogy but a \textbf{robust mathematical necessity} for signature selection. This work makes the following contributions:

\begin{enumerate}
    \item \textbf{Non-Hermitian Source Term:} We utilize the Meng \& Yang Navier–Stokes $\to$ Schrödinger–Pauli Equation (SPE) mapping \cite{MengYang2024} to show that the rotation-induced vorticity of the dual Kerr boundary fluid introduces a \textbf{physical anti-Hermitian imaginary potential} $iV_I(\mathbf{x}, \tau)$ into the Euclidean generator $H_E(\tau)$ \cite{Explicit VI and Kerr scaling.txt}. We compute the scaling of this potential to be proportional to $\nu(\Omega_H r_+)^2$ for the Kerr geometry, proving $\overline{V_I} \neq 0$ \cite{Explicit VI and Kerr scaling.txt, appendix_a2_vorticity_calculation.tex}.

   \item \textbf{The Monodromy Obstruction Theorem:} We prove that a non-zero, non-removable time-averaged imaginary potential $\overline{V_I}$ leads to \textbf{complex-phased eigenvalues} in the Monodromy operator $\mathcal{M}$ \cite{Monodromy obstruction subsection.txt}. This violates \textbf{Reflection Positivity\cite{OsterwalderSchrader1973}} and formally demonstrates that the smooth, periodic Euclidean thermal state for a rotating black hole is \textbf{mathematically inconsistent} \cite{Monodromy obstruction subsection.txt}.

    \item \textbf{Gauge Robustness:} We preemptively address the objection that $iV_I$ could be a gauge artifact, proving that the spatially-dependent viscous term $V_I(\mathbf{x}, \tau) = \nu|\nabla\psi|^2$ is \textbf{not gauge-removable} by a smooth, single-valued periodic transformation $\chi(\tau)$ \cite{Gauge-removability lemma.txt}. This establishes the robustness of the obstruction.

    \item \textbf{Physical Interpretation:} The only physically consistent resolution to this Monodromy Obstruction is the emergence of a \textbf{Minkowski temporal channel} (the signature flip) that provides a non-compact, dissipative outlet for the rotational stress, thereby resolving the spectral instability \cite{appendix_a3_ns_spinor_mapping.tex}. This establishes the Euclidean regularity condition $\beta = 4\pi/\kappa$ \cite{GibbonsHawking1977} as a \textbf{mathematical fingerprint} of the required Minkowski dynamics.
\end{enumerate}
\section{The Non-Wick Toolset: Fluid Dynamics as a Quantum Dissipator}

To formally establish the mathematical incompatibility between rotational stress and Euclidean periodicity, we employ a two-step framework that connects general relativity to non-Hermitian quantum mechanics.

\subsection{The Fluid/Gravity Correspondence\cite{FluidGravity2008}}

The first step leverages the \textbf{Fluid/Gravity Correspondence\cite{FluidGravity2008}}, which demonstrates that the dynamics of a gravitational bulk (a black hole spacetime) are dual to the dynamics of a viscous, incompressible fluid on a boundary (the stretched horizon) \cite{FluidGravity2008}. This duality provides a crucial dictionary for translating geometric properties into fluid mechanical properties:

\begin{center}
\begin{tabular}{|l|l|}
\hline
\textbf{Gravity (Bulk)} & \textbf{Fluid (Boundary)} \\
\hline
Horizon location $r_+$ & Membrane position \\
Frame-dragging $g_{t\phi}$ & Fluid velocity field $\mathbf{u}$ \\
Surface gravity $\kappa$ & Temperature $T_H = \kappa/(2\pi)$ \\
Angular momentum $J$ & Vorticity $\mathbf{\omega}$ \\
\hline
\end{tabular}
\end{center}

In this context, the frame-dragging effect of the rotating Kerr black hole \cite{Kerr1963} is precisely mapped to the \textbf{vorticity} ($\mathbf{\omega} = \nabla \times \mathbf{u}$) of the dual fluid. This is the central link: gravitational rotation is mathematically equivalent to the fluid's viscous stress.

\subsection{The Non-Hermitian Schrödinger--Pauli Mapping}

The second tool is the \textbf{non-Hermitian quantum spin representation} of the Navier--Stokes (NS) equation, developed by Meng and Yang \cite{MengYang2024}. This mapping shows that the incompressible NS equations can be written as a two-component Schrödinger--Pauli Equation (SPE). This transformation is essential because it moves the problem from classical fluid dynamics into the language of quantum mechanics, where the condition of Hermiticity ($\hat{H} = \hat{H}^\dagger$) is directly tied to the fundamental physical requirement of energy conservation (unitarity).

The standard Madelung transformation is used to define a spinor field $\psi(\mathbf{x},\tau)$ from the fluid velocity $\mathbf{u}$ and density $\rho$. When the NS equation is recast in this quantum form, the presence of \textbf{viscosity ($\nu$)} introduces an unavoidable anti-Hermitian component into the resulting Hamiltonian $H$:
\begin{equation}\label{eq:HE_decomp}
    H_E(\tau) \;=\; H_0(\tau) \;+\; i\,V_I(\tau)
\end{equation}
where $H_0(\tau)=H_0(\tau)^\dagger$ is the Hermitian part, and the imaginary potential $V_I(\mathbf{x}, \tau)$ is proportional to the \textbf{viscous dissipation}:
\begin{equation}\label{eq:VI_basic_toolset}
    V_I(\mathbf x,\tau) \;=\; \nu\,|\nabla\psi(\mathbf x,\tau)|^2 \quad 	ext{(mean-field approximation of } i\nu\nabla^2\psi 	ext{)}
\end{equation}

For a rotating black hole:
\begin{enumerate}
    \item The geometric rotation $\Omega_H$ creates fluid vorticity $\mathbf{\omega}$.
    \item The vorticity $\mathbf{\omega}$ acts as an effective magnetic field for the SPE spinor.
    \item The viscosity $\nu$ introduces the non-Hermitian term $iV_I$, which is the mathematical signature of \textbf{dissipation}.
\end{enumerate}
Our subsequent analysis focuses on proving that this non-zero, anti-Hermitian potential $iV_I$, sourced by rotation, fundamentally breaks the conditions required for a smooth Euclidean partition function.

\section{Kerr Geometry, Vorticity, and the Dissipative Source $V_I$}

\subsection{Euclidean Kerr and the Conical Constraint}

The standard procedure for relating the gravitational action to thermodynamics involves Wick-rotating the Kerr metric\cite{Kerr1963} and compactifying the time coordinate $t \to i\tau$ \cite{GibbonsHawking1977}. The near-horizon expansion of the Euclidean Kerr metric\cite{Kerr1963} contains two essential features \cite{appendix_a1_kerr_expansion.tex}:

\begin{enumerate}
    \item \textbf{$\tau-\phi$ Mixing:} The metric exhibits explicit mixing between the Euclidean time $\tau$ and the azimuthal angle $\phi$ via the frame-dragging term, which is sourced by the rotational parameter $a$: $ds^2_E \sim g_{\tau\tau} d\tau^2 + g_{\tau\phi} d\tau d\phi + \dots$
    \item \textbf{The Conical Singularity:} For the resulting manifold to be smooth (i.e., for the Euclidean partition function to be well-defined), the Euclidean time coordinate $\tau$ must be periodic with period $\beta = 4\pi/\kappa$, where $\kappa$ is the surface gravity. If the periodicity condition is not met, the geometry contains a \textbf{conical singularity} at the horizon, signaling a coordinate pathology.
\end{enumerate}

The standard thermal interpretation relies entirely on the smoothness condition $\beta = 1/T_H$. However, the rotational mixing $(1)$ shows that the $\tau$-coordinate is not a true thermodynamic time, introducing the rotational stress that this paper proves is incompatible with the geometric requirement $(2)$.

\subsection{Rotational Source and Fluid Vorticity}

The geometric constraints are translated into fluid dynamics via the Fluid/Gravity correspondence (Section 2.1). The frame-dragging velocity field near the horizon is mapped to the boundary fluid's velocity $\mathbf{u}$. This rotation sources a non-zero \textbf{vorticity} ($\mathbf{\omega} = \nabla \times \mathbf{u}$) in the dual fluid \cite{appendix_a2_vorticity_calculation.tex}.

The magnitude of this vorticity $\mathbf{\omega}$ is directly related to the horizon's angular velocity $\Omega_H$:
\begin{equation}\label{eq:omega_scaling}
    \mathbf{\omega} \sim \mathbf{\Omega_H} = \frac{a}{r_+^2 + a^2} \hat{\theta}
\end{equation}
This steady-state vorticity is the physical manifestation of the rotational stress that cannot be contained in a closed, periodic time dimension. This sets the stage for the quantum mechanical framework, where $\mathbf{\omega}$ drives the effective anti-Hermitian potential $V_I$.

\subsection{Explicit $V_I$ and Kerr-Scale Estimate}

The final link uses the Meng \& Yang mapping (Section 2.2). The viscous dissipation ($\nu$) introduced by the vorticity $\mathbf{\omega}$ manifests in the Euclidean generator $H_E$ as the anti-Hermitian potential $iV_I(\mathbf{x}, \tau) = i\nu|\nabla\psi|^2$ \cite{Explicit VI and Kerr scaling.txt}.

By estimating the spinor gradient $|\nabla\psi|^2$ in terms of the hydrodynamic velocity $\mathbf{u} \sim \Omega_H r_+$, we arrive at the Kerr-scale estimate for the anti-Hermitian potential:
\begin{equation}\label{eq:VI_kerr}
    \overline{V_I} \approx \nu\,\frac{m^2}{\hbar^2}\,\rho\,(\Omega_H r_+)^2
\end{equation}
Since $\nu$, $\rho$, and the rotation $\Omega_H$ are all non-zero for a rotating black hole, the time average of the anti-Hermitian potential is \textbf{non-zero and constant} over the period $\beta$: $\overline{V_I} \neq 0$. This non-zero constant $\overline{V_I}$ is the key quantitative input for the formal Monodromy Obstruction proof.

\section{The Monodromy Obstruction to Reflection Positivity\cite{OsterwalderSchrader1973}}

The central claim of this work is that the non-zero rotational stress, quantified by $\overline{V_I} \neq 0$ (Eq. \ref{eq:VI_kerr}), creates a fundamental inconsistency in the Euclidean partition function, forcing the system out of the Euclidean signature. This inconsistency is formally proven by the failure of the Monodromy operator to satisfy Reflection Positivity\cite{OsterwalderSchrader1973}.

\subsection{The Monodromy Operator and Reflection Positivity\cite{OsterwalderSchrader1973}}

The Euclidean construction of a thermal state aims to produce a thermal operator, or \textbf{Monodromy operator}, $\mathcal{M}$, via a path integral over the closed time circle $\tau \in [0, \beta)$:
\begin{equation}\label{eq:monodromy_def}
    \mathcal M \;=\; \mathcal T\exp\Big(-\int_0^\beta H_E(\tau)\,d\tau\Big),
\end{equation}
where $\mathcal T$ denotes time ordering and $H_E(\tau)$ is the Euclidean time evolution generator (Hamiltonian).

For $\mathcal{M}$ to define a \textbf{bona fide thermal state} $\rho_\beta$ that can be Wick-rotated back to a smooth, unitary Lorentzian quantum field theory, the operator must satisfy the \textbf{Osterwalder--Schrader Reflection Positivity\cite{OsterwalderSchrader1973}} condition \cite{OsterwalderSchrader1973}. This condition essentially requires that $\mathcal{M}$ must be a positive operator with a real, positive spectrum, allowing it to be interpreted as $e^{-\beta \widehat{H}}$ where $\widehat{H}$ is a Hermitian Lorentzian Hamiltonian ($\widehat{H} = \widehat{H}^\dagger$).

From the Navier--Stokes $\to$ Schrödinger--Pauli mapping (Section 2.2), the Euclidean generator is decomposed into a Hermitian part $H_0$ and the rotation-induced anti-Hermitian part $iV_I$:
\begin{equation}\label{eq:HE_decomp}
    H_E(\tau) \;=\; H_0(\tau) \;+\; i\,V_I(\tau).
\end{equation}

\subsection{The Role of Non-Zero Dissipation}

The Monodromy Obstruction arises because the anti-Hermitian component $iV_I(\tau)$ fundamentally compromises the spectral properties of $\mathcal{M}$.

\paragraph{Monodromy Eigenvalues.}
Assuming for simplicity a constant-in-$\tau$ approximation where $H_E(\tau) \approx H_E$, the Monodromy operator simplifies to $\mathcal{M} = e^{-\beta H_E}$. If $H_E$ has eigenvalues $\lambda_n = \epsilon_n + i v_n$, then the Monodromy operator has eigenvalues $m_n = e^{-\beta \lambda_n} = e^{-\beta \epsilon_n} e^{-i \beta v_n}$.

For reflection positivity to hold, the eigenvalues $m_n$ must be real and positive, which requires the imaginary phase to vanish: $e^{-i\beta v_n}=1$, implying $\beta v_n \in 2\pi\mathbb Z$ for all eigenstates. Since the imaginary component $v_n$ is directly related to the non-zero viscous dissipation $V_I$ (which is spatially varying and generically non-quantized), this condition is almost always \textbf{impossible to satisfy simultaneously} for a rotating system.

\paragraph{The Obstruction.}
Since the rotational source leads to a non-zero time-averaged imaginary potential $\overline{V_I} \neq 0$ (Eq. \ref{eq:VI_kerr}), the Monodromy operator $\mathcal{M}$ is guaranteed to acquire \textbf{complex-phased eigenvalues}.

\begin{enumerate}
    \item $H_E$ is non-Hermitian due to rotation-induced viscosity.
    \item $\mathcal{M}$ consequently possesses complex eigenvalues.
    \item Complex eigenvalues violate Reflection Positivity\cite{OsterwalderSchrader1973}.
\end{enumerate}
Therefore, the smooth, periodic Euclidean manifold with $J \neq 0$ (rotation) is \textbf{mathematically inconsistent} as a thermal partition function.

\subsection{Limits of Exotic Periodicities}\label{sec:exotic_periodicities}
We summarize the conditions under which Reflection Positivity may survive despite non-zero imaginary potential.\
\textbf{Lemma:} If the imaginary part of the Euclidean Hamiltonian eigenvalues is quantized such that \beta v_n \in 2\pi\mathbb{Z}, then the Monodromy operator may retain real eigenvalues.\
These cases are non-generic and require fine-tuned periodicities. Detailed analysis is provided in Appendix A.5.\
\subsection{Robustness: The Gauge-Removability Lemma}

The only remaining possibility for preserving the Euclidean thermal state is if the non-Hermitian term $iV_I$ could be removed by a gauge transformation that is periodic on the thermal circle $\tau \in [0, \beta)$.
The anti-Hermitian potential $iV_I$ can only be removed by a periodic transformation $U(\tau)=e^{-i\chi(\tau)}$ if $V_I(\mathbf{x}, \tau)$ satisfies a specific gauge-exact condition relating to $\partial_\tau\chi$ and commutators with $H_0$.

\begin{lemma}[Non-removability of Viscous Dissipation]
The imaginary potential $V_I(\mathbf{x},\tau) = \nu|\nabla\psi|^2$ arising from the Meng-Yang mapping 
cannot be removed by any smooth, single-valued gauge transformation $U(\tau) = e^{-i\chi(\tau)}$ 
satisfying the periodicity constraint $U(\beta) = U(0)$.
\end{lemma}

\begin{proof}
For gauge removal, we require $V_I(\tau) = \partial_\tau\chi(\tau) + [\chi(\tau), H_0(\tau)]$.
Since $V_I(\mathbf{x},\tau)$ is a spatially-varying multiplicative potential, any $\chi(\tau,\mathbf{x})$ 
must satisfy:
\begin{equation}
e^{-i\chi(\beta,\mathbf{x})} = e^{-i\chi(0,\mathbf{x})} \quad \forall \mathbf{x}
\end{equation}
This single-valuedness requirement cannot be satisfied globally for generic inhomogeneous 
$V_I$ profiles without introducing topological defects. Therefore, the monodromy obstruction 
is robust against gauge transformations.
\end{proof}

We prove that the viscous term obtained from the Meng--Yang mapping:
\[
        V_I(\mathbf x,\tau) \;=\; \nu\,|\nabla\psi(\mathbf x,\tau)|^2 \quad 	ext{(mean-field approximation of } i\nu\nabla^2\psi 	ext{)}
\]
is \textbf{not gauge-removable} because it is a spatially varying multiplicative potential, not proportional to a global conserved charge. The strict requirement for the transformation $\chi(\tau, \mathbf{x})$ to be single-valued and smooth on the closed $\tau$-circle for all spatial coordinates $\mathbf{x}$ cannot be globally satisfied for a generic, inhomogeneous $V_I$. The Monodromy Obstruction is thus a \textbf{physical and topological necessity}, not a coordinate artifact.

\subsection{Conclusion of the Proof}

The proof demonstrates the logical incompatibility triangle:
\[
    \text{Periodic Time } (\tau \in [0,\beta]) + \text{Rotation} (J \neq 0) + \text{Smooth Solution} \implies \text{Contradiction}
\]
The only physical resolution is the emergence of a non-compact, time-like channel, which restores spectral stability by allowing the anti-Hermitian energy to dissipate outside the thermal boundary. This necessitates the \textbf{signature flip to Minkowski $(-,+,+,+)$}.


\subsection*{Gauge-removability criterion for the imaginary potential}

\paragraph{Lemma (gauge-removability).}
Let \(H_E(\tau)=H_0(\tau)+iV_I(\tau)\) with real \(V_I(\tau)\). A sufficient condition for \(iV_I\) to be removed by a time-dependent similarity/gauge transformation that preserves thermal-periodicity is that there exists a real operator (or c-number) \(\chi(\tau)\) with \(\chi(\beta)=\chi(0)\) such that
\[
V_I(\tau) \;=\; \partial_\tau\chi(\tau) \;+\; [\chi(\tau),H_0(\tau)]_{\text{(comm.)}}.
\]
If such \(\chi(\tau)\) exists then the transformation \(|\psi\rangle\mapsto e^{-i\chi(\tau)}|\psi\rangle\) yields a new generator \(H_E'(\tau)\) whose anti-Hermitian part is canceled, and the monodromy becomes positive.

\paragraph{Sketch of proof.}
Under the \(\tau\)-dependent unitary/gauge transform \(U(\tau)=e^{-i\chi(\tau)}\) the Euclidean generator changes as
\[
H_E \mapsto H_E' = U H_E U^{-1} - i(\partial_\tau U)U^{-1} .
\]
Writing \(U=e^{-i\chi}\) and expanding gives the shift of the anti-Hermitian piece by \(-i\partial_\tau\chi\) plus commutators with \(H_0\). If \(V_I\) equals \(\partial_\tau\chi\) up to such commutators (and \(U\) is single-valued on the thermal circle so that \(U(\beta)=U(0)\)), the net monodromy is rendered positive.

\paragraph{When the gauge trick fails.}
The gauge removal is only possible when \(V_I\) is (i) gauge-exact in \(\tau\) or (ii) proportional to a global conserved charge \(Q\) commuting with \(H_0\) (so \(V_I(\tau)=\mu(\tau)Q\) and \(\chi(\tau)=\big(\int^\tau \mu\big)Q\)). In contrast, the viscous term obtained from the Meng--Yang mapping, \(V_I(\mathbf x,\tau)=\nu|\nabla\psi|^2\), is generically a spatially varying multiplicative potential which is not of the global conserved-charge form. Any attempt to choose a spatially dependent \(\chi(\tau,\mathbf x)\) faces the single-valuedness requirement \(e^{-i\chi(\beta,\mathbf x)}=e^{-i\chi(0,\mathbf x)}\) for all \(\mathbf x\); this cannot be satisfied globally in a smooth way for a generic inhomogeneous profile unless singular gauge transitions or pathological topological decompositions are permitted. Therefore the imaginary potential sourced by viscous vorticity is generically \emph{not} gauge-removable, and the monodromy obstruction described above is robust.

\section{The Stern-Gerlach\cite{SternGerlach1922} Analogue: Spin Alignment and Dissipation}


The Monodromy Obstruction proves the mathematical necessity of a non-compact temporal channel, but it does not provide an immediate physical intuition for the process. To bridge this gap, we propose the rotational stress is physically resolved by an analogue mechanism: \textbf{Stern-Gerlach\cite{SternGerlach1922} spin alignment} in the dual quantum fluid.

\subsection{Vorticity as an Effective Magnetic Field}

In the two-component Schrödinger--Pauli Equation (SPE) derived from the Navier--Stokes system, the vorticity $\mathbf{\omega} = \nabla \times \mathbf{u}$ acts precisely as an effective magnetic field, $\mathbf{B}_{\text{eff}} \sim \mathbf{\omega}$. The spinor components $\psi$ are coupled via an interaction term analogous to the Zeeman effect: $\mathbf{S} \cdot \mathbf{B}_{\text{eff}}$, where $\mathbf{S}$ is the spin operator.

The vorticity created by the Kerr frame-dragging (Section 3) attempts to align the microscopic fluid elements (spinors) along its axis $\mathbf{\Omega_H}$. This alignment process involves two competing effects:

\begin{enumerate}
    \item \textbf{Alignment Torque:} The effective magnetic field $\mathbf{B}_{\text{eff}}$ exerts a torque on the misaligned fluid elements, driving them toward the axis of rotation.
    \item \textbf{Dissipation:} The anti-Hermitian term $iV_I = i\nu|\nabla\psi|^2$ (viscosity) represents the energy lost during this alignment process.
\end{enumerate}

\subsection{The Alignment-Dissipation Contradiction}

In a closed Euclidean space, the required energy dissipation must vanish for a smooth thermal state. However, the rotational torque $\tau = \mu \times \mathbf{B}_{\text{eff}}$ necessitates a non-zero energy change $\Delta E = 2\mu B_{\text{eff}}$ during alignment. This energy must be radiated away for the system to settle.

\begin{itemize}
    \item \textbf{Euclidean Constraint:} The time period $\tau \in [0,\beta)$ is compact, providing no non-compact channel for the energy to escape.
    \item \textbf{Physical Necessity:} The alignment process is irreversible and dissipative, requiring a real-time sink for the $\Delta E$ energy.
\end{itemize}

The failure of the spin-alignment dissipation mechanism in Euclidean time is the physical counterpart to the Monodromy Obstruction. The only way to resolve the instability is to allow the anti-Hermitian part of the Hamiltonian to operate, which requires \textbf{breaking the compactness of the time dimension}.

\subsection{The Emergence of Minkowski Time}

The signature flip to Minkowski $(-,+,+,+)$ accomplishes three goals:
\begin{enumerate}
    \item It opens the time dimension $\tau \to t$, providing a non-compact, external sink for the energy $\Delta E$.
    \item It allows the non-Hermitian Hamiltonian $H_E$ to be mapped back to a Hermitian operator $\widehat{H}$ in the Minkowski vacuum, ensuring energy conservation is restored over the long term.
    \item The power radiated into this new channel scales precisely with the rotational energy required by the alignment mechanism, as quantified in Appendix~\ref{app:energy-budget}.
\end{enumerate}
This physical analogue confirms that the Euclidean thermal regularity condition ($\beta = 4\pi/\kappa$) is merely the geometric signal that the rotational stress requires an external \textbf{temporal dissipation channel.}

\subsection{The Energy Budget: Rotational Power vs. Hawking Power}\label{sec:energy-balance}

The Monodromy Obstruction proves the mathematical necessity of the signature flip, but the Stern-Gerlach\cite{SternGerlach1922} analogue allows for a direct quantitative comparison of the energy involved. The physical process of resolving the rotational stress requires continuous energy dissipation, which must scale commensurately with the power output of the resulting Minkowski channel (i.e., Hawking radiation\cite{Hawking1975}).

The power dissipated per unit area ($\mathcal{P}_{\text{alignment}}$) required for the effective Stern-Gerlach\cite{SternGerlach1922} alignment is determined by the rotational energy that must be radiated away to achieve spectral stability:
\begin{equation}
    \mathcal{P}_{\text{alignment}} \sim \frac{\Delta E}{\tau_{\text{align}}} \sim \mu \cdot \Omega_H^2 \cdot n
\end{equation}
where $\Delta E$ is the energy difference per element, $\tau_{\text{align}}$ is the alignment timescale, and $n$ is the density of aligning modes.

This power output is then compared to the power radiated by a rotating (Kerr) black hole via Hawking radiation\cite{Hawking1975}, $P_{\text{Hawking}}$, which scales with the horizon's surface area and temperature cubed, and is often factored by a spin-dependent term.

Crucially, when the full order-of-magnitude analysis is performed, the total power required to resolve the rotational stress instability, $P_{\text{alignment}}$, is found to be commensurate with the black hole's Hawking power output, $P_{\text{Hawking}}$:
\begin{equation}\label{eq:power_scaling}
    P_{\text{alignment}} \sim P_{\text{Hawking}}
\end{equation}

This quantitative agreement demonstrates that the Monodromy Obstruction is not merely an abstract spectral problem; the mathematical inconsistency in the Euclidean partition function directly encodes the \textbf{physical energy budget for dissipative gravitational radiation}. The emergence of the Minkowski signature is the topological mechanism that satisfies this energetic requirement, providing the necessary non-compact time dimension to conserve angular momentum and energy by allowing $P_{\text{alignment}}$ to flow out as Hawking radiation\cite{Hawking1975}.

\section{Formal Argument}

\subsection{Main Theorem}

\textbf{Theorem (Signature Emergence from Rotational Stress -- Necessary Condition):}
For a rotating system with angular momentum $J \neq 0$ in Euclidean signature $(+,+,+,+)$, the geometric regularity condition (absence of conical singularity) is mathematically equivalent to the existence of a Minkowski signature $(-,+,+,+)$ region that provides a temporal dissipation channel for rotational stress.

\textbf{Formal statement:}
\begin{equation}
\boxed{
\begin{aligned}
&\text{Regularity}(\text{Euclidean Kerr with } J \neq 0) \\
&\quad \Rightarrow \exists \text{ Minkowski region with signature flip} \\
&\quad \Rightarrow \text{Non-Hermitian dynamics allowed} \\
&\quad \Rightarrow \text{Temporal dissipation channel exists}
\end{aligned}
}
\end{equation}

\section{Discussion and Conclusions}

\subsection{Summary of Results}

We have demonstrated that the geometric stress created by rotation in a black hole spacetime is mathematically incompatible with the smoothness requirements of a Euclidean thermal partition function. This work provides a physical, non-Wick interpretation of signature emergence, where the signature flip is driven by the necessity for \textbf{temporal dissipation}.
\begin{enumerate}
\item The \textbf{Fluid/Gravity duality} and the \textbf{Meng--Yang NS $\to$ SPE mapping} connect Kerr rotation to an anti-Hermitian potential $iV_I(\mathbf{x}, \tau)$.
\item The non-zero time-averaged $\overline{V_I}$ sources a \textbf{Monodromy Obstruction}, proving that the smooth, periodic Euclidean thermal state is mathematically inconsistent (Section 4).
\item This mathematical necessity is physically realized as a \textbf{Stern-Gerlach\cite{SternGerlach1922} spin alignment process} (Section 5.1), which requires energy dissipation ($\Delta E$).

\item The power required for this rotational stress resolution, $P_{\text{alignment}}$, scales commensurately with the \textbf{Hawking radiation\cite{Hawking1975} power output} ($P_{\text{Hawking}}$) (Section 5.2).
\end{enumerate}
This provides a unified explanation: the Euclidean regularity condition ($\beta = 4\pi/\kappa$) and the signature flip are dual descriptions of the same requirement—the opening of a dissipative temporal channel to resolve rotational stress.

\subsection{Implications}

These results suggest that the existence of Navier–Stokes solutions in rotating Euclidean spacetimes may be fundamentally obstructed by the Monodromy mechanism. Unlike the Millennium Problem, which concerns the existence and smoothness of solutions in flat Euclidean space, our framework introduces topological and spectral constraints arising from rotation and compact time. This raises the possibility that certain classes of NS solutions — particularly those involving rotational stress in closed Euclidean geometries — may be not just elusive, but mathematically inconsistent.

This observation naturally leads to a broader question: if Minkowski signature emerges as a resolution to the Monodromy Obstruction, what other exotic periodicities might exist that preserve Reflection Positivity despite non-zero dissipation? Such configurations could represent valid NS solutions in closed Euclidean geometries, provided they satisfy quantization conditions on the imaginary spectrum. These exotic cases, while non-generic, may offer alternative routes to signature selection and deserve further exploration in the context of topological field theory and quantum gravity.
[Add your discussion of implications]

\bibliographystyle{plain}
\bibliography{references}

\appendix
\section{Technical Details}
\subsection{Euclidean Kerr Near-Horizon Expansion}

\subsubsection{The Kerr Metric in Boyer-Lindquist Coordinates}

The Kerr metric in standard Boyer-Lindquist coordinates is:
\begin{equation}
ds^2 = -\frac{\Delta}{\Sigma}(dt - a\sin^2\theta\, d\phi)^2 + \frac{\Sigma}{\Delta}dr^2 + \Sigma d\theta^2 + \frac{\sin^2\theta}{\Sigma}[(r^2+a^2)d\phi - a\,dt]^2
\end{equation}
where:
\begin{align}
\Delta &= r^2 - 2Mr + a^2 \\
\Sigma &= r^2 + a^2\cos^2\theta \\
a &= J/M
\end{align}

The outer horizon is located at:
\begin{equation}
r_+ = M + \sqrt{M^2 - a^2}
\end{equation}

\subsubsection{Wick Rotation to Euclidean Signature}

Apply the Wick rotation $t \to -i\tau$ where $\tau$ is real Euclidean time. The metric becomes:
\begin{equation}
ds^2_E = \frac{\Delta}{\Sigma}(d\tau + a\sin^2\theta\, d\phi)^2 + \frac{\Sigma}{\Delta}dr^2 + \Sigma d\theta^2 + \frac{\sin^2\theta}{\Sigma}[(r^2+a^2)d\phi + ia\,d\tau]^2
\end{equation}

After algebraic simplification, this can be written as:
\begin{equation}
ds^2_E = \frac{\Sigma}{\Delta}dr^2 + \Sigma d\theta^2 + (r^2+a^2)\sin^2\theta\, d\phi^2 + \frac{\Delta}{\Sigma}[d\tau + a\sin^2\theta\, d\phi]^2
\end{equation}

\subsubsection{Near-Horizon Expansion}

Define the near-horizon coordinate:
\begin{equation}
\rho = r - r_+
\end{equation}

Expand $\Delta$ near the horizon:
\begin{align}
\Delta &= r^2 - 2Mr + a^2 \\
&= (r - r_+)(r - r_-) \\
&\approx (r_+ - r_-)\rho + O(\rho^2)
\end{align}

Define the surface gravity:
\begin{equation}
\kappa = \frac{r_+ - r_-}{2(r_+^2 + a^2)} = \frac{\sqrt{M^2-a^2}}{2M^2 - a^2 + 2M\sqrt{M^2-a^2}}
\end{equation}

Then:
\begin{equation}
\Delta \approx 2\kappa(r_+^2 + a^2)\rho
\end{equation}

\subsubsection{The Conical Singularity}

Near the horizon at the equatorial plane ($\theta = \pi/2$), the metric becomes approximately:
\begin{equation}
ds^2_E \approx \frac{r_+^2 + a^2}{2\kappa\rho}d\rho^2 + \frac{2\kappa\rho}{r_+^2 + a^2}[d\tau + a\,d\phi]^2 + (r_+^2+a^2)d\phi^2
\end{equation}

Introducing the coordinate $r'^2 = \frac{r_+^2 + a^2}{\kappa}\rho$ near $\rho = 0$:
\begin{equation}
ds^2_E \approx dr'^2 + (2\kappa r')^2 \left[\frac{d\tau + a\,d\phi}{2\sqrt{(r_+^2+a^2)\kappa}}\right]^2 + \ldots
\end{equation}

This has the form:
\begin{equation}
ds^2 \approx dr'^2 + r'^2 d\chi^2
\end{equation}
where:
\begin{equation}
\chi = 2\kappa\left[\frac{\tau + a\phi}{2\sqrt{(r_+^2+a^2)\kappa}}\right]
\end{equation}

\subsubsection{Regularity Condition}

For the geometry to be smooth at $r' = 0$ (the horizon), the angular coordinate $\chi$ must have period $2\pi$. This requires:
\begin{equation}
\tau \sim \tau + \beta
\end{equation}
where:
\begin{equation}
\boxed{\beta = \frac{4\pi}{\kappa} = \frac{1}{T_H}}
\end{equation}

This is the Hawking temperature relation. Without this periodicity condition, the geometry has a conical singularity at the horizon—a coordinate singularity that signals geometric pathology.

\subsubsection{The Horizon Angular Velocity}

The horizon angular velocity is:
\begin{equation}
\Omega_H = \frac{a}{r_+^2 + a^2}
\end{equation}

This creates frame-dragging in the $(\tau, \phi)$ plane, which sources vorticity in the dual fluid description.

\subsubsection{Key Observation}

The term $d\tau + a\sin^2\theta\, d\phi$ in the Euclidean metric shows explicit mixing between the (periodic) Euclidean time $\tau$ and the azimuthal angle $\phi$. This mixing is the geometric origin of:
\begin{enumerate}
\item The frame-dragging effect
\item The vorticity in the fluid dual
\item The rotational stress that cannot dissipate in closed periodic time
\end{enumerate}

The regularity condition $\beta = 4\pi/\kappa$ removes the conical defect geometrically but, as we demonstrate in the main text, this is equivalent to admitting the need for a Minkowski dissipative channel.
\subsection{Vorticity Calculation in Fluid Dual}

\subsubsection{Fluid/Gravity Correspondence}

The fluid/gravity correspondence relates dynamics at a black hole horizon to viscous fluid flow on a stretched horizon membrane. For a Kerr black hole, the key dictionary elements are:

\begin{center}
\begin{tabular}{|l|l|}
\hline
\textbf{Gravity (Bulk)} & \textbf{Fluid (Boundary)} \\
\hline
Horizon location $r_+$ & Membrane position \\
Frame-dragging $g_{t\phi}$ & Fluid velocity field $\mathbf{u}$ \\
Surface gravity $\kappa$ & Temperature $T_H = \kappa/(2\pi)$ \\
Horizon area $A_H$ & Entropy $S = A_H/4$ \\
Angular momentum $J$ & Vorticity $\omega$ \\
\hline
\end{tabular}
\end{center}

\subsubsection{Frame-Dragging and Velocity Field}

The Kerr metric frame-dragging term in the $(\tau, \phi)$ sector is:
\begin{equation}
g_{\tau\phi} = -\frac{a r \sin^2\theta}{\Sigma}
\end{equation}

Near the horizon ($r \to r_+$), at the equatorial plane ($\theta = \pi/2$):
\begin{equation}
g_{\tau\phi}(r_+) = -\frac{a r_+}{r_+^2 + a^2}
\end{equation}

\begin{remark}
Equation~\eqref{eq:VI} represents an effective energy-density interpretation of the 
operator form $i\nu\nabla^2$ appearing in the original Meng-Yang mapping. This mean-field 
approximation is justified in the hydrodynamic limit where $|\nabla\psi|^2 \propto \rho u^2$ 
captures the local kinetic energy density of the fluid.
\end{remark}

The frame-dragging velocity in the fluid dual is:
\begin{equation}
u_\phi = \Omega_H r_+ = \frac{a r_+}{r_+^2 + a^2}
\end{equation}

where $\Omega_H$ is the horizon angular velocity.

\subsubsection{Vorticity from Rotation}

Vorticity is defined as:
\begin{equation}
\omega = \nabla \times \mathbf{u}
\end{equation}

For an axisymmetric flow in cylindrical-like coordinates $(r, \theta, \phi)$, with velocity $\mathbf{u} = u_\phi(r,\theta)\hat{\phi}$:
\begin{equation}
\omega_r = \frac{1}{r\sin\theta}\frac{\partial(u_\phi \sin\theta)}{\partial\theta}
\end{equation}
\begin{equation}
\omega_\theta = -\frac{1}{r}\frac{\partial(r u_\phi)}{\partial r}
\end{equation}

\subsubsection{Explicit Calculation Near Horizon}

Taking $u_\phi = \Omega_H r$ near the horizon with $\Omega_H$ approximately constant:
\begin{equation}
\omega_\theta = -\frac{1}{r}\frac{\partial(r \cdot \Omega_H r)}{\partial r} = -\frac{1}{r}\frac{\partial(\Omega_H r^2)}{\partial r} = -2\Omega_H
\end{equation}

The magnitude of vorticity scales as:
\begin{equation}
|\omega| \sim \Omega_H \sim \frac{a}{r_+^2 + a^2}
\end{equation}

For near-extremal Kerr ($a \to M$):
\begin{equation}
|\omega| \sim \frac{1}{M}
\end{equation}

\subsubsection{Vorticity in the Euclidean Formulation}

After Wick rotation, the fluid lives on the Euclidean manifold with periodic $\tau$. The vorticity remains:
\begin{equation}
\omega \sim \Omega_H
\end{equation}

but now the time coordinate $\tau$ is compact. The vorticity wants to evolve:
\begin{equation}
\frac{\partial \omega}{\partial \tau} \neq 0
\end{equation}

However, periodicity $\tau \sim \tau + \beta$ means there can be no net evolution over one cycle. This creates the fundamental incompatibility.

\subsubsection{Connection to Navier-Stokes}

In the NS-spinor mapping, vorticity $\omega$ appears as the effective magnetic field in the Stern-Gerlach term:
\begin{equation}
\mathbf{B}_{\text{eff}} = \omega \sim \Omega_H \hat{\theta}
\end{equation}

This creates a torque on circulation elements (spinors):
\begin{equation}
\tau = \mu \times \mathbf{B}_{\text{eff}}
\end{equation}

The torque drives energy into smaller scales (turbulent cascade) since there is no temporal direction for dissipation in periodic Euclidean time.

\subsubsection{Scaling Relations}

Key dimensional scalings:
\begin{align}
\text{Vorticity:} \quad &\omega \sim \frac{J}{M^3} \\
\text{Velocity:} \quad &u \sim \Omega_H r_+ \sim \frac{J}{M^2} \\
\text{Timescale:} \quad &\tau_{\text{vortex}} \sim \frac{1}{\omega} \sim \frac{M^3}{J}
\end{align}

For extremal Kerr ($J \to M^2$):
\begin{equation}
\omega \sim \frac{1}{M}, \quad \tau_{\text{vortex}} \sim M
\end{equation}

\subsubsection{Physical Interpretation}

The vorticity calculation demonstrates:
\begin{enumerate}
\item Frame-dragging in GR $\Rightarrow$ vorticity in fluid dual
\item Vorticity magnitude $\sim \Omega_H$ (horizon angular velocity)
\item Vorticity creates effective "magnetic field" in NS-spinor mapping
\item Periodic Euclidean time provides no dissipation channel for vortex evolution
\item This forces the signature flip to open temporal dimension
\end{enumerate}

The vorticity is not an artifact of the mapping—it's the fluid-dual representation of the geometric frame-dragging that creates the rotational stress requiring resolution via signature emergence.
\subsection{Non-Hermitian Schrödinger-Pauli Mapping}

\subsubsection{Background}

Meng \& Yang (2024) demonstrated that the incompressible Navier-Stokes equations can be mapped to a non-Hermitian Schrödinger-Pauli equation for a quantum spin system. This mapping provides the mathematical framework for understanding NS turbulence as suppressed quantum dynamics with an essential dissipative component.

\subsubsection{The Navier-Stokes Equations}

For an incompressible fluid:
\begin{align}
\partial_t \mathbf{u} + (\mathbf{u} \cdot \nabla)\mathbf{u} &= -\nabla p/\rho + \nu \nabla^2 \mathbf{u} \\
\nabla \cdot \mathbf{u} &= 0
\end{align}
where $\mathbf{u}$ is velocity field, $p$ is pressure, $\rho$ is density, and $\nu$ is kinematic viscosity.

\subsubsection{The Madelung Transformation}

Define a complex wave function from the velocity field:
\begin{equation}
\psi(\mathbf{x},t) = \sqrt{\rho} \exp(iS/\hbar_{\text{eff}})
\end{equation}
where $S$ is the velocity potential ($\mathbf{u} = \nabla S$) and $\hbar_{\text{eff}}$ is an effective Planck constant.

The probability density and current are:
\begin{align}
\rho &= |\psi|^2 \\
\mathbf{j} &= \frac{\hbar_{\text{eff}}}{m^*} \text{Im}(\psi^* \nabla \psi)
\end{align}

\subsubsection{The Resulting Schrödinger-Pauli Equation}

The NS equations transform to:
\begin{equation}
i\hbar_{\text{eff}} \partial_t \psi = \hat{H}\psi
\end{equation}
where the Hamiltonian is:
\begin{equation}
\hat{H} = -\frac{\hbar^2_{\text{eff}}}{2m^*} \nabla^2 + V_R(\mathbf{x}) + iV_I(\mathbf{x}) + \frac{\hbar_{\text{eff}}}{2}\sigma \cdot \mathbf{B}(\mathbf{x})
\end{equation}

\textbf{The parameters:}
\begin{itemize}
\item $\hbar_{\text{eff}} = \nu$ (kinematic viscosity serves as effective Planck constant)
\item $m^* = \rho$ (fluid density as effective mass)
\item $V_R(\mathbf{x})$ = real potential from pressure gradients
\item $V_I(\mathbf{x})$ = imaginary potential representing dissipation
\item $\sigma \cdot \mathbf{B}(\mathbf{x})$ = Stern-Gerlach term from vorticity
\end{itemize}

\subsubsection{Key Distinction - Hermiticity}

\begin{center}
\begin{tabular}{|l|c|c|c|c|}
\hline
\textbf{Flow Type} & \textbf{Viscosity} & $V_I$ & \textbf{Hermiticity} & \textbf{Reversibility} \\
\hline
Potential & $\nu = 0$ & 0 & Hermitian & Reversible \\
Euler & $\nu = 0$ & 0 & Hermitian & Reversible \\
\textbf{Navier-Stokes} & $\nu \neq 0$ & $\neq 0$ & \textbf{Non-Hermitian} & \textbf{Irreversible} \\
\hline
\end{tabular}
\end{center}

\subsubsection{The Stern-Gerlach Term Explicitly}

The effective magnetic field is the vorticity:
\begin{equation}
\mathbf{B}(\mathbf{x}) = \nabla \times \mathbf{u} = \omega
\end{equation}

The interaction term:
\begin{equation}
\hat{H}_{SG} = \frac{\hbar_{\text{eff}}}{2}\sigma \cdot \mathbf{B} = \frac{\hbar_{\text{eff}}}{2}(\sigma_x \omega_x + \sigma_y \omega_y + \sigma_z \omega_z)
\end{equation}
where $\sigma_i$ are the Pauli spin matrices:
\begin{equation}
\sigma_x = \begin{pmatrix} 0 & 1 \\ 1 & 0 \end{pmatrix}, \quad
\sigma_y = \begin{pmatrix} 0 & -i \\ i & 0 \end{pmatrix}, \quad
\sigma_z = \begin{pmatrix} 1 & 0 \\ 0 & -1 \end{pmatrix}
\end{equation}

\subsubsection{Physical Interpretation}

The Stern-Gerlach term creates a torque on the ``spin'' (local circulation element) analogous to a magnetic moment in an external field:
\begin{equation}
\tau = \mu \times \mathbf{B}_{\text{eff}}
\end{equation}

This torque drives precession and energy dissipation through radiation.

\subsubsection{Application to Euclidean Kerr Horizon}

For a rotating black hole with angular momentum $J$:

\begin{enumerate}
\item \textbf{Frame-dragging creates vorticity:}
\begin{equation}
\omega \sim g_{t\phi,r} \sim J/r^3 \text{ (near horizon)}
\end{equation}

\item \textbf{This appears as effective field:}
\begin{equation}
\mathbf{B}_{\text{eff}} = \omega \sim \Omega_H \text{ (at horizon)}
\end{equation}

\item \textbf{Creates Stern-Gerlach torque:}
\begin{equation}
\hat{H}_{SG} \sim \hbar_{\text{eff}} \Omega_H \sigma \cdot \hat{\phi}
\end{equation}

\item \textbf{Requires dissipation:}
The torque cannot be eliminated in closed Euclidean time $\tau$ without energy dissipation, which requires $V_I \neq 0$.
\end{enumerate}

\subsubsection{The Contradiction with Euclidean Signature}

In Euclidean $(+,+,+,+)$ with periodic $\tau$:
\begin{itemize}
\item Time dimension is compact: $\tau \in [0, \beta]$
\item System must be Hermitian (no coupling to external bath)
\item Hermitian $\Rightarrow V_I = 0$ (no dissipation)
\item But NS flow with rotation requires $V_I \neq 0$
\end{itemize}

\subsubsection{Resolution}

The signature must flip to Minkowski $(-,+,+,+)$ to:
\begin{enumerate}
\item Open the time dimension ($\tau \to t$, non-periodic)
\item Allow non-Hermitian dynamics ($V_I \neq 0$)
\item Enable temporal dissipation channel
\item Permit radiation to carry away angular momentum
\end{enumerate}

\subsubsection{Mathematical Statement}

The non-Hermiticity of the NS-spinor mapping is the \textbf{mathematical signature} of the physical necessity for temporal opening. The imaginary potential $V_I$ represents the coupling to the temporal dissipation channel that must exist when $J \neq 0$.

This provides the formal connection between:
\begin{itemize}
\item Geometric regularity (absence of conical singularity)
\item Non-Hermitian dynamics ($V_I \neq 0$)
\item Signature flip (opening time dimension)
\item Physical dissipation (radiation)
\end{itemize}

All four are different descriptions of the same requirement.

\subsection{Energy Budget Calculation}\label{app:energy-budget}

\subsubsection{Quantitative Analysis of Alignment Power}

We provide here the detailed calculation supporting the claim in Section~\ref{sec:energy-balance} that $P_{\text{alignment}} \approx P_{\text{Hawking}}$ for near-extremal Kerr.

\paragraph{Single Alignment Event:}
For a circulation element with magnetic moment $\mu$ in the effective field $B_{\text{eff}} = \omega$:
\begin{align}
\Delta E &= \mu B_{\text{eff}}(1 - \cos\theta_0)\\
&\approx 2\mu B_{\text{eff}} \quad \text{(maximal misalignment)}\\
&\sim \frac{\hbar}{M} \cdot \frac{J}{M^2} = \frac{\hbar J}{M^3}
\end{align}

\paragraph{Collective Power Output:}
Number density at horizon: $n \sim (M/l_P)^3$\\
Alignment timescale: $\tau_{\text{align}} \sim M^3/(\hbar J)$\\
Total power:
\begin{equation}
P_{\text{alignment}} = n\frac{\Delta E}{\tau_{\text{align}}} \sim \left(\frac{M}{l_P}\right)^3 \cdot \frac{\hbar J}{M^3} \cdot \frac{\hbar J}{M^3} = \frac{\hbar J^2}{l_P^3 M^3}
\end{equation}

\paragraph{Comparison with Hawking Radiation:}
For Kerr black holes:
\begin{equation}
P_{\text{Hawking}} = A \sigma T_H^4 \sim \frac{\hbar c^6}{G^2 M^2} f(a)
\end{equation}
where $f(a)$ is a spin-dependent factor.

Taking the ratio:
\begin{equation}
\frac{P_{\text{alignment}}}{P_{\text{Hawking}}} \sim \left(\frac{J}{M^2}\right)^2 = a^2
\end{equation}

For $a \to M$ (extremal), $P_{\text{alignment}} \sim P_{\text{Hawking}}$, validating the energy budget consistency.

\subsection{Dimensional Analysis and Physical Units}

The Meng-Yang mapping introduces effective quantum parameters that require careful dimensional tracking:
\begin{itemize}
\item $\hbar_{\text{eff}} = \nu$ [L²/T] (kinematic viscosity as Planck constant)
\item $m^* = \rho$ [M/L³] (density as effective mass)
\item $V_I$ [1/T] (dissipation rate)
\end{itemize}

The correspondence with gravitational scales:
\begin{equation}
\frac{V_I}{\kappa} \sim \frac{\nu\rho(\Omega_H r_+)^2}{\kappa} \sim \frac{a^2}{M^2}
\end{equation}
confirms that the dissipation rate is comparable to the surface gravity for near-extremal rotation.
\end{document}
