\subsection{Euclidean Kerr Near-Horizon Expansion}

\subsubsection{The Kerr Metric in Boyer-Lindquist Coordinates}

The Kerr metric in standard Boyer-Lindquist coordinates is:
\begin{equation}
ds^2 = -\frac{\Delta}{\Sigma}(dt - a\sin^2\theta\, d\phi)^2 + \frac{\Sigma}{\Delta}dr^2 + \Sigma d\theta^2 + \frac{\sin^2\theta}{\Sigma}[(r^2+a^2)d\phi - a\,dt]^2
\end{equation}
where:
\begin{align}
\Delta &= r^2 - 2Mr + a^2 \\
\Sigma &= r^2 + a^2\cos^2\theta \\
a &= J/M
\end{align}

The outer horizon is located at:
\begin{equation}
r_+ = M + \sqrt{M^2 - a^2}
\end{equation}

\subsubsection{Wick Rotation to Euclidean Signature}

Apply the Wick rotation $t \to -i\tau$ where $\tau$ is real Euclidean time. The metric becomes:
\begin{equation}
ds^2_E = \frac{\Delta}{\Sigma}(d\tau + a\sin^2\theta\, d\phi)^2 + \frac{\Sigma}{\Delta}dr^2 + \Sigma d\theta^2 + \frac{\sin^2\theta}{\Sigma}[(r^2+a^2)d\phi + ia\,d\tau]^2
\end{equation}

After algebraic simplification, this can be written as:
\begin{equation}
ds^2_E = \frac{\Sigma}{\Delta}dr^2 + \Sigma d\theta^2 + (r^2+a^2)\sin^2\theta\, d\phi^2 + \frac{\Delta}{\Sigma}[d\tau + a\sin^2\theta\, d\phi]^2
\end{equation}

\subsubsection{Near-Horizon Expansion}

Define the near-horizon coordinate:
\begin{equation}
\rho = r - r_+
\end{equation}

Expand $\Delta$ near the horizon:
\begin{align}
\Delta &= r^2 - 2Mr + a^2 \\
&= (r - r_+)(r - r_-) \\
&\approx (r_+ - r_-)\rho + O(\rho^2)
\end{align}

Define the surface gravity:
\begin{equation}
\kappa = \frac{r_+ - r_-}{2(r_+^2 + a^2)} = \frac{\sqrt{M^2-a^2}}{2M^2 - a^2 + 2M\sqrt{M^2-a^2}}
\end{equation}

Then:
\begin{equation}
\Delta \approx 2\kappa(r_+^2 + a^2)\rho
\end{equation}

\subsubsection{The Conical Singularity}

Near the horizon at the equatorial plane ($\theta = \pi/2$), the metric becomes approximately:
\begin{equation}
ds^2_E \approx \frac{r_+^2 + a^2}{2\kappa\rho}d\rho^2 + \frac{2\kappa\rho}{r_+^2 + a^2}[d\tau + a\,d\phi]^2 + (r_+^2+a^2)d\phi^2
\end{equation}

Introducing the coordinate $r'^2 = \frac{r_+^2 + a^2}{\kappa}\rho$ near $\rho = 0$:
\begin{equation}
ds^2_E \approx dr'^2 + (2\kappa r')^2 \left[\frac{d\tau + a\,d\phi}{2\sqrt{(r_+^2+a^2)\kappa}}\right]^2 + \ldots
\end{equation}

This has the form:
\begin{equation}
ds^2 \approx dr'^2 + r'^2 d\chi^2
\end{equation}
where:
\begin{equation}
\chi = 2\kappa\left[\frac{\tau + a\phi}{2\sqrt{(r_+^2+a^2)\kappa}}\right]
\end{equation}

\subsubsection{Regularity Condition}

For the geometry to be smooth at $r' = 0$ (the horizon), the angular coordinate $\chi$ must have period $2\pi$. This requires:
\begin{equation}
\tau \sim \tau + \beta
\end{equation}
where:
\begin{equation}
\boxed{\beta = \frac{4\pi}{\kappa} = \frac{1}{T_H}}
\end{equation}

This is the Hawking temperature relation. Without this periodicity condition, the geometry has a conical singularity at the horizon—a coordinate singularity that signals geometric pathology.

\subsubsection{The Horizon Angular Velocity}

The horizon angular velocity is:
\begin{equation}
\Omega_H = \frac{a}{r_+^2 + a^2}
\end{equation}

This creates frame-dragging in the $(\tau, \phi)$ plane, which sources vorticity in the dual fluid description.

\subsubsection{Key Observation}

The term $d\tau + a\sin^2\theta\, d\phi$ in the Euclidean metric shows explicit mixing between the (periodic) Euclidean time $\tau$ and the azimuthal angle $\phi$. This mixing is the geometric origin of:
\begin{enumerate}
\item The frame-dragging effect
\item The vorticity in the fluid dual
\item The rotational stress that cannot dissipate in closed periodic time
\end{enumerate}

The regularity condition $\beta = 4\pi/\kappa$ removes the conical defect geometrically but, as we demonstrate in the main text, this is equivalent to admitting the need for a Minkowski dissipative channel.