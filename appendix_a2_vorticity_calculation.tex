\subsection{Vorticity Calculation in Fluid Dual}

\subsubsection{Fluid/Gravity Correspondence}

The fluid/gravity correspondence relates dynamics at a black hole horizon to viscous fluid flow on a stretched horizon membrane. For a Kerr black hole, the key dictionary elements are:

\begin{center}
\begin{tabular}{|l|l|}
\hline
\textbf{Gravity (Bulk)} & \textbf{Fluid (Boundary)} \\
\hline
Horizon location $r_+$ & Membrane position \\
Frame-dragging $g_{t\phi}$ & Fluid velocity field $\mathbf{u}$ \\
Surface gravity $\kappa$ & Temperature $T_H = \kappa/(2\pi)$ \\
Horizon area $A_H$ & Entropy $S = A_H/4$ \\
Angular momentum $J$ & Vorticity $\omega$ \\
\hline
\end{tabular}
\end{center}

\subsubsection{Frame-Dragging and Velocity Field}

The Kerr metric frame-dragging term in the $(\tau, \phi)$ sector is:
\begin{equation}
g_{\tau\phi} = -\frac{a r \sin^2\theta}{\Sigma}
\end{equation}

Near the horizon ($r \to r_+$), at the equatorial plane ($\theta = \pi/2$):
\begin{equation}
g_{\tau\phi}(r_+) = -\frac{a r_+}{r_+^2 + a^2}
\end{equation}

\begin{remark}
Equation~\eqref{eq:VI} represents an effective energy-density interpretation of the 
operator form $i\nu\nabla^2$ appearing in the original Meng-Yang mapping. This mean-field 
approximation is justified in the hydrodynamic limit where $|\nabla\psi|^2 \propto \rho u^2$ 
captures the local kinetic energy density of the fluid.
\end{remark}

The frame-dragging velocity in the fluid dual is:
\begin{equation}
u_\phi = \Omega_H r_+ = \frac{a r_+}{r_+^2 + a^2}
\end{equation}

where $\Omega_H$ is the horizon angular velocity.

\subsubsection{Vorticity from Rotation}

Vorticity is defined as:
\begin{equation}
\omega = \nabla \times \mathbf{u}
\end{equation}

For an axisymmetric flow in cylindrical-like coordinates $(r, \theta, \phi)$, with velocity $\mathbf{u} = u_\phi(r,\theta)\hat{\phi}$:
\begin{equation}
\omega_r = \frac{1}{r\sin\theta}\frac{\partial(u_\phi \sin\theta)}{\partial\theta}
\end{equation}
\begin{equation}
\omega_\theta = -\frac{1}{r}\frac{\partial(r u_\phi)}{\partial r}
\end{equation}

\subsubsection{Explicit Calculation Near Horizon}

Taking $u_\phi = \Omega_H r$ near the horizon with $\Omega_H$ approximately constant:
\begin{equation}
\omega_\theta = -\frac{1}{r}\frac{\partial(r \cdot \Omega_H r)}{\partial r} = -\frac{1}{r}\frac{\partial(\Omega_H r^2)}{\partial r} = -2\Omega_H
\end{equation}

The magnitude of vorticity scales as:
\begin{equation}
|\omega| \sim \Omega_H \sim \frac{a}{r_+^2 + a^2}
\end{equation}

For near-extremal Kerr ($a \to M$):
\begin{equation}
|\omega| \sim \frac{1}{M}
\end{equation}

\subsubsection{Vorticity in the Euclidean Formulation}

After Wick rotation, the fluid lives on the Euclidean manifold with periodic $\tau$. The vorticity remains:
\begin{equation}
\omega \sim \Omega_H
\end{equation}

but now the time coordinate $\tau$ is compact. The vorticity wants to evolve:
\begin{equation}
\frac{\partial \omega}{\partial \tau} \neq 0
\end{equation}

However, periodicity $\tau \sim \tau + \beta$ means there can be no net evolution over one cycle. This creates the fundamental incompatibility.

\subsubsection{Connection to Navier-Stokes}

In the NS-spinor mapping, vorticity $\omega$ appears as the effective magnetic field in the Stern-Gerlach term:
\begin{equation}
\mathbf{B}_{\text{eff}} = \omega \sim \Omega_H \hat{\theta}
\end{equation}

This creates a torque on circulation elements (spinors):
\begin{equation}
\tau = \mu \times \mathbf{B}_{\text{eff}}
\end{equation}

The torque drives energy into smaller scales (turbulent cascade) since there is no temporal direction for dissipation in periodic Euclidean time.

\subsubsection{Scaling Relations}

Key dimensional scalings:
\begin{align}
\text{Vorticity:} \quad &\omega \sim \frac{J}{M^3} \\
\text{Velocity:} \quad &u \sim \Omega_H r_+ \sim \frac{J}{M^2} \\
\text{Timescale:} \quad &\tau_{\text{vortex}} \sim \frac{1}{\omega} \sim \frac{M^3}{J}
\end{align}

For extremal Kerr ($J \to M^2$):
\begin{equation}
\omega \sim \frac{1}{M}, \quad \tau_{\text{vortex}} \sim M
\end{equation}

\subsubsection{Physical Interpretation}

The vorticity calculation demonstrates:
\begin{enumerate}
\item Frame-dragging in GR $\Rightarrow$ vorticity in fluid dual
\item Vorticity magnitude $\sim \Omega_H$ (horizon angular velocity)
\item Vorticity creates effective "magnetic field" in NS-spinor mapping
\item Periodic Euclidean time provides no dissipation channel for vortex evolution
\item This forces the signature flip to open temporal dimension
\end{enumerate}

The vorticity is not an artifact of the mapping—it's the fluid-dual representation of the geometric frame-dragging that creates the rotational stress requiring resolution via signature emergence.