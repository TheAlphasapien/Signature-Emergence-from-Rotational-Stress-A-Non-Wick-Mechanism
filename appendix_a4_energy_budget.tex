\subsection{Energy Budget for Stern-Gerlach Alignment}

\subsubsection{Physical Setup}

Consider a magnetic moment $\mu$ in the effective field $\mathbf{B}_{\text{eff}} = \omega$ created by frame-dragging at the Kerr horizon.

\paragraph{Initial state (misaligned):}
\begin{itemize}
\item Angle $\theta$ between $\mu$ and $\mathbf{B}_{\text{eff}}$
\item Torque: $\tau = \mu \times \mathbf{B}_{\text{eff}} = \mu B_{\text{eff}} \sin(\theta)$
\item Rotational energy: $E_{\text{rot}} = \mu B_{\text{eff}}(1 - \cos(\theta))$
\end{itemize}

\paragraph{Final state (aligned):}
\begin{itemize}
\item $\theta = 0$ (fully aligned)
\item $E_{\text{rot}} = 0$
\item Energy difference radiated away
\end{itemize}

\subsubsection{Energy Radiated per Alignment Event}

\begin{equation}
\Delta E = \mu B_{\text{eff}}(1 - \cos(\theta_0))
\end{equation}

For maximum misalignment ($\theta_0 = \pi$):
\begin{equation}
\Delta E = 2\mu B_{\text{eff}}
\end{equation}

\subsubsection{Order of Magnitude Estimate for Black Hole Horizon}

Effective magnetic field from frame-dragging:
\begin{equation}
B_{\text{eff}} \sim \omega \sim \Omega_H \sim J/M^2 \text{ (for Kerr)}
\end{equation}

Effective magnetic moment (from quantum of circulation):
\begin{equation}
\mu \sim \hbar/M \text{ (quantum angular momentum per unit mass)}
\end{equation}

Therefore:
\begin{equation}
\Delta E \sim \frac{\hbar}{M} \cdot \frac{J}{M^2} \sim \frac{\hbar J}{M^3} \sim \frac{\hbar \Omega_H}{M}
\end{equation}

\paragraph{For a solar mass black hole ($M \sim M_\odot$):}
\begin{equation}
\Delta E \sim 10^{-40} \text{ J per alignment event}
\end{equation}

\subsubsection{Total Energy Flux}

Number density of ``circulation quanta'' at horizon:
\begin{equation}
n \sim (M/l_P)^3 \sim 10^{76} \text{ per horizon area}
\end{equation}

Alignment timescale (from Larmor precession):
\begin{equation}
\tau_{\text{align}} \sim \frac{1}{\mu B_{\text{eff}}} \sim \frac{M^3}{\hbar J} \sim \frac{M^2}{\hbar a}
\end{equation}

For near-extremal Kerr ($a \to M$):
\begin{equation}
\tau_{\text{align}} \sim \frac{M^2}{\hbar} \sim \frac{\beta}{2\pi}
\end{equation}

This directly connects the alignment timescale to the thermal periodicity $\beta = 1/T_H$ required by Euclidean regularity.

\subsubsection{Power Radiated}

\begin{equation}
P \sim n \frac{\Delta E}{\tau_{\text{align}}} \sim \frac{J^2}{M^4}
\end{equation}

\subsubsection{Comparison to Hawking Radiation}

\begin{equation}
P_{\text{Hawking}} \sim \frac{\hbar c^6}{G^2 M^2} \sim \frac{1}{M^2}
\end{equation}

The ratio:
\begin{equation}
\frac{P_{\text{alignment}}}{P_{\text{Hawking}}} \sim \left(\frac{J}{M}\right)^2 \sim a^2
\end{equation}

For maximally rotating Kerr ($a \to M$):
\begin{equation}
P_{\text{alignment}} \sim P_{\text{Hawking}}
\end{equation}

\subsubsection{Physical Interpretation}

The energy budget shows that:
\begin{enumerate}
\item Individual alignment events are quantum-scale
\item Collective effect produces macroscopic radiation
\item Power scales with spin parameter squared
\item Comparable to Hawking radiation for fast rotation
\item Radiation pattern is \textbf{cylindrically symmetric} (key signature!)
\end{enumerate}

\subsubsection{Observational Signature}

For astrophysical black holes with strong magnetic fields:
\begin{itemize}
\item Enhanced radiation in equatorial plane
\item Jet alignment with spin axis
\item Timescale for alignment $\sim 10^3$--$10^6$ $M$ (seconds to hours for stellar mass)
\end{itemize}

This matches observed jet formation timescales in X-ray binaries.

\subsubsection{Significance for Signature Emergence}

The agreement between alignment power and Hawking power (up to the spin-dependent factor $a^2$) demonstrates that the Stern-Gerlach mechanism is not merely analogous but captures the actual energy budget required by the geometric regularity condition. The fact that the alignment timescale $\tau_{\text{align}}$ approaches the thermal periodicity $\beta$ for near-extremal rotation provides further quantitative support. This quantitative agreement supports the interpretation of signature emergence as a physical symmetry-breaking process rather than a mathematical artifact.